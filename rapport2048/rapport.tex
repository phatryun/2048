\documentclass[11pt,a4paper]{article}
\usepackage[utf8]{inputenc}
\usepackage[T1]{fontenc}
\usepackage{lmodern}
\usepackage[scale=0.8]{geometry}
\usepackage[francais]{babel}
\usepackage{amsmath}
\usepackage{amsfonts}
\usepackage{amssymb}
\usepackage{graphicx}
\usepackage{xcolor}


\begin{document}

\begin{center}
HUREL Arnaud\\
Elève-Ingénieur en master  Informatique : Système Intelligent      
\vspace{3cm}



{\textbf{\Huge{\\Métaheuristique et jeu\\}}}
\vspace{1cm}


{\textbf{\Large{Projet de construction d'une IA du jeu 2048\\}}}
\vspace{4cm}

\textsc{{\large{A l'attention de Monsieur CAZENAVE\\}}}


\vspace{4cm}
À Paris le \today
\end{center}
\newpage

\tableofcontents
\newpage

\section{Introduction sur le 2048}
Le jeu 2048 a été créer en 2014 par Gabriele Cirulli à l'age de 19 ans, le but de ce jeu est de faire glisser des tuiles sur un grille de 4x4 cases afin d'obtenir une tuile égale à 2048. En effet, les tuilles ont pour valeur des puissance de 2. A chaque tour, le jeu fait apparaitre de manière aléatoire soit une tuile 2, soit une tuile 4 (90\% pour la tuile 2 et 10\% pour la tuile 4). Pour jouer, l'utilisateur a le choix entre quatres coup : Nord, Sud, Est et Ouest. Ceux-ci auront, respectivement, un effet de comprésion des tuiles vers le haut, le bas, la droite et la gauche.

Le jeu ainsi que les IAs ont été développer en python, vous pouvez y acceder en toute liberté sur mon Github à l'adresse suivante : https://github.com/phatryun/2048


\section{Algorithmes utilisés}
Une fois le jeu créé, nous avons mis en place différents algortithme en vu de créer un intelligence artificiel capable de résoudre se casse tête. Nous avons alors regarder sur internet afin de se renseigner sur les différents algorithme utilisé

Pour des raisons techniques, lors de nos test nous avons choisi de faire tourner nos algortimes que avec une profondeur de 4.

\subsection{Algorithme Naif}
Cet algorithme consiste à déterminer le meilleur coup parmis tous ceux possible. Il peut être considéré comme naif, car il vas tout simplement effectuer tous les coups de manière exhaustif sans logique particulière. Dans notre fichier source il corresdond aux fonctions \textit{nextMove} et \textit{nextMoveRecur}.\\
Voici son derroulement : 
\begin{itemize}
\item Tant que la profondeur de recherche n'est pas égale à celle voulut, on va effectuer chaque coup possible (Nord, Sud, Est, Ouest)
\item On va, ensuite, ajouter de manière aléatoire la nouvelle tuile pour ensuite calculer le score heuristique de la grille obtenue.
\item Puis on va étudier ces fils qui lui retourneront leur meilleur score calculé par l'heuristique choisit. 
\item De ces dernier, on retiendra le coup pour le quel le score de l'heuristique à été le plus élévè.
\item Enfin on va ajouter à son propore score celui de son meilleur fils mais de façon pondéré. 
\end{itemize}

\subsection{Expectimax}

Cet algorithme est répartie en deux étapes : la première consiste à rechercher le meilleur coup possible calculé en maximisant le score obtenu grâce à une heuristique et ensuite de minimiser l'impacte de l'ajout aléatoire des nouvelles tuiles.
Dans notre code nous le retrouverons à travers la fonction \textit{player\_max} qui aura pour objectid de maximiser le choix du coup et la fonction \textit{player\_expect} qui 
va minimiser les placements de la nouvelle tuile.\\
Voici plus en détail leurs déroulement :

\begin{itemize}
\item Premièrement, notre condition d'arret sera quand la profondeur de recherche sera égale à 0. Alors ont calculera notre score grâce à notre heuristique.
\item La première étape consiste à tester tous les coup possible du moment qu'ils sont autorisé.
\item Si ils le sont, alors on va passer à la deuxième partis de l'algo qui va minimiser le positionnement aléatoire de nos nouvelles tuiles
\item Pour cela on va repérer tous les emplacements vide de notre grille et pour chaque emplacement on va ajouter une tuile pour repartir dans la première étape (calcul du score max).
\item De ces score obtenue on va effectuer un calcul de score moyen pour obtenir une estimation de score en fonction de toutes les positions de nouvelles tuile possible
\end{itemize}


\section{Heuristiques utilisé}

\subsection{Utilisation du score}
La première heuristique qu'y nous est venue à l'esprit est celle maximisant le score de la grille. En effet, celui ci est automatiquement calculé à chaque fusion de tuile. Il fut donc facile de mettre en place cette heuristique qui va préférer le coup rapportant le plus de point. Malheureusement cette heuristique n'est pas tres concluante (comme le prouve les résultats obtenues ci-dessous). En effet, celle-ci va préférer les fusions de tuiles direct au défaut des fusions qui pourraient se produire au coup d'après et qui rapporteraient plus de point.

\subsection{Gradient}
L'heuristique du gradient permet de favoriser les grilles qui ont pour valeur maximale des tuille de dans un coin de la grille. Ainsi pour chaques grilles, on va calculer sont score en fonction des 4 gradients et on retournera le meilleur score. Le score est le résultat d'une somme des multiplication entre les valeurs des tuiles et les poids contenu dans la matrice des gradients.\\
Cette heurisitque est facile à mettre en place mais elle est très couteuse car en effet, pour chaque appel de cette fonction on va calculer 4 score. Néanmois on obtient de bon résultats

\subsection{Ordre des valeurs des cases}
Cet heuristique permet de preferer le choix de construction d'une suitre parfaite. En effet, avec un système de pondération des valeurs de la grille, nous allons définir une suite parfaite qui pour notre cas commancera en bas à droite de notre grille.

\subsection{Conjugaison d'heuristiques avec ajout de la maximisation des cases libres}
De ces deux dernières heuristiques, nous avons également rajouté un impacte sur le nombre de case vide. En effet, nous allons dans ce cas maximiser le nombre de case vide restant après le coup dans le but de faire durer le plus longtemps possible la partie.

\newpage
\section{Résultats obtenus}

Les résultats obtenus ont été réalisés sur une machine Debian 64bit doté de 4 processeurs Intel Core i5-2520M CPU \@ 2.50GHz et de 8Go de RAM. Pour des raisons technique, nous avons réalisé les test avec comme profondeur maximum de 4. Dans les tableaux ci-desous nous observons la fréquence d'atteinte de palier sur 25 tests.

\begin{tabular}{|l|c|c|c|c|c|} 
   \hline
    Algorithme & \multicolumn{5}{c|}{ Naif } \\
    \hline
    Heuristique & Score & Gradiant & Gradient ++ & ordre & ordre ++ \\
    \hline
    0 & 13 & 3 & 0 & 4 & 0\\
    \hline
    1024 & 12 & 5 & 0 & 1 & 0\\
    \hline
    2024 & 0 & 13 & 0 & 13 & 0 \\
    \hline
    4096 & 0 & 4 & 0 & 7 & 0 \\
    \hline
    8192 & 0 & 0 & 0 & 0 & 0 \\
    \hline
    temps moyen & 0 & 239 & 0 & 209 & 0 \\
    \hline
\end{tabular}



\vspace{2cm}


\begin{tabular}{|l|c|c|c|c|c|} 
   \hline
    Algorithme & \multicolumn{5}{c|}{ Expectimax }\\
    \hline
    Heuristique & Score & Gradiant & Gradient ++ & ordre & ordre ++ \\
    \hline
    0 & 25 & 10 & 0 & 2 & 0\\
    \hline
    1024 & 0 & 13 & 0 & 3 & 0\\
    \hline
    2024 & 0 & 2 & 0 & 18 & 0 \\
    \hline
    4096 & 0 & 0 & 0 & 2 & 0 \\
    \hline
    8192 & 0 & 0 & 0 & 0 & 0 \\
    \hline
    temps moyen & 32 & 183 & 0 & 94 & 0 \\
    \hline
\end{tabular}


\end{document}